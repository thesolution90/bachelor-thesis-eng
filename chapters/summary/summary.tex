\newpage
\section{Summary and outlook} \label{toc:zusammenfassungundausblick}

\subsection{Hypothesis testing} \label{toc:ueberpruefungderhypothesen}

In the present thesis it was answered whether it is possible by means of application of a \ac{BI} process to optimize
mining data centers. To be able to answer this question, the first step in chapter
\ref{toc:grundlagenbusinessintelligence} and \ref{toc:grundlagenkryptomining} was the basics for understanding
business intelligence, mining of cryptocurrencies and the financial \acp{KPI} of mining data centers were laid.
This knowledge is used as the basis for the subsequent part (chapter \ref{toc:ansatzmoeglichkeitenfuerbusinessintelligence})
used to conduct a deductive reasoning that gives a first answer to the main hypothesis.
This reasoning is divided into several steps. In the first step, the available data sources were identified,
These were analyzed for suitability towards a \ac{BI} Process. Subsequently
all relevant \acp{KPI} were identified for each sub-hypothesis. The result of this analysis was that
the data sources are suitable and sufficient for the realization of a \ac{BI} Process. In the next step
various analysis procedures were listed that could be considered for processing the data.
For this purpose, these procedures were briefly described and possible examples and approaches within the sub-hypotheses were
identified. As a result, it can be determined that the analysis methods for the case of descriptive
\ac{BI} are sufficient. For predictive models, an answer is not possible at this stage. In the following chapter
(chapter \ref{toc:planungeinesbiprozessesfuereinminingrechenzentrum}) a case study was carried out on the basis of the so far elaborated
results, which took place within the Genesis Group. This case study
examines the internal feasibility of the previous results and complements them. In the first step
a stakeholder analysis was carried out. This is used for the planning of an in-house \ac{BI} Process.
Subsequently, the individual steps of the process were analyzed and possible software products identified.
In addition, a focus was placed on the presentation of the data, as this has not yet been discussed in the previous argumentation.
Finally, the case study was analyzed and a result for the main hypothesis was formed.

On the basis of the argumentation and the case study, it can be determined that the partial hypotheses and thus also the main hypothesis are
true. It is possible that data centers that mine cryptocurrencies can be financially optimized by applying a \ac{BI}
Process.

However, limitations exist with the chosen approach within this thesis. These include the following points:
\begin{itemize}
    \item Ultimately, the \ac{BI} Process in the present work has been planned in a theoretical context. Whether such a
    process actually works in reality can only be clarified by putting this planning into practice.
    However, the probability that this will not work is very low, since the case study
    should cover the practical implementation as best as possible.
    \item The various forms of analysis have been pointed out throughout the paper. However, it was decided not to
    concrete elaboration of these forms of analysis and their implementation. Therefore, only an implementation of the
    analysis algorithms can say whether these meet the requirements.
    \item The integration of a \ac{BI} Process was discussed only on the basis of the available data sources. In the absence
    or existence of additional data sources, the design of the \ac{BI} process may change in a way that is not
    been discussed in this paper. However, the data sources that have been pointed out are with a
    high probability available in any company that participates in mining. Even when using other
    types of mining hardware, the parameters may change leading to a change in the \ac{BI} process.
    \item Consensus algorithms other than \ac{PoW} were not discussed in this paper. These have to be individually adapted
    into a \ac{BI} process.
\end{itemize}

Despite these points, it is possible to consider the main hypothesis as answered, since all data sources used in this work are available in any company on the market.
Only in special cases small adjustments have to be made, but these will not change the overall conclusion of this work.

\subsection{Further research approaches} \label{toc:ausblick}

From the limitations identified in chapter \ref{toc:ueberpruefungderhypothesen}, further
research approaches can be derived. This thesis focused on \ac{BI} in the environment of the Bitcoin Blockchain and its
relevant technologies. However, other use cases of blockchain technology also exist, and their investigation in the
\ac{BI} context can be interesting. These include the following technologies, among others:

\begin{itemize}
    \item \textbf{\ac{DeFi}: }In the course of this work, cryptocurrency exchanges were used as a data source. These are by nature
    inherently centralized services provided by individual companies, such as Binance\footnote{https://www.binance.com/en}
    or Kraken\footnote{https://www.kraken.com}. Developers are already working on the decentralized equivalent of these
    services, referred to as \ac{DeFi}.\footcite[Cf.][]{wiesflecker2021trends} Also, there will be
    this technology introduces a credit system in the cryptocurrency space, which may be of particular interest to companies
    that need liquidity in cryptocurrencies.\footcite[Cf.][]{wiesflecker2021trends} The
    Exploration of the role of \ac{DeFi} in \ac{BI} is therefore a current example of a research gap building on this work.
    \item \textbf{Smart Contracts: }For some time now, so-called smart contracts have also been mapped using Blockchain
    Technology. The best-known example of this implementation is the Ethereum
    Blockchain.\footcite[Cf.][]{wiesflecker2021trends} Smart Contracts are decentralized equivalents of classical contracts
    within a blockchain.\footcite[Cf.][]{wiesflecker2021trends} As especially contracts with customers, service providers
    and suppliers are interesting for \ac{BI} systems, therefore, exploring the decentralized counterpart would be a
    gap in which research can still be done.
    \item \textbf{\acp{NFT}: }These tokens represent unique values.\footcite[Cf.][]{wiesflecker2021trends} 
    In this regard, such tokens may represent, for example, relevant investments by companies in art, instruments
    and other unique investment assets. How and whether these interact with the field of \ac{BI} can be the subject of
    of a further work on this topic. 
    \item \textbf{Other consensus algorithms and currencies: }As described at the beginning of this work, only the Bitcoin
    Blockchain and the \ac{PoW} consensus algorithm were analyzed within this thesis. Since there are other consensus algorithms,
    such as \ac{PoS}, the investigation of further cryptocurrencies and consensus algorithms would be of
    interest. Through such an investigation, this work can be suitably complemented and
    the topic of \ac{BI} and cryptocurrencies to be explored in more depth.
\end{itemize}

Based on the veracity of the main hypothesis of this work, these aspects can deepen the possible uses of \ac{BI} in the
cryptomining context in more depth and identify further potential use cases.
