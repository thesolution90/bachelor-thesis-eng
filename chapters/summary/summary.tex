\newpage
\section{Summary and outlook} \label{toc:zusammenfassungundausblick}

\subsection{Hypothesis testing} \label{toc:ueberpruefungderhypothesen}

In der vorliegenden Arbeit wurde beantwortet, ob es mittels Anwendung eines \ac{BI} Prozesses möglich ist,
Mining Rechenzentren zu optimieren. Um diese Frage beantworten zu können, wurden im ersten Schritt in Kapitel
\ref{toc:grundlagenbusinessintelligence} und \ref{toc:grundlagenkryptomining} die Grundlagen für das Verständnis
von Business Intelligence, Mining von Kryptowährungen und die finanziellen \acp{KPI} von Mining Rechenzentren
gelegt. Dieses Wissen wird als Grundlage für den darauf folgenden Teil (Kapitel \ref{toc:ansatzmoeglichkeitenfuerbusinessintelligence})
verwendet, um eine deduktive Argumentation zu betreiben, die eine erste Antwort auf die Haupthypothese gibt.
Diese Argumentation teilt sich in mehrere Schritte auf. Im ersten Schritt wurden die Datenquellen identifiziert,
die zur Verfügung stehen. Diese wurden auf eine Eignung für einen \ac{BI} Prozess hin analysiert. Folgend
wurden alle relevanten \acp{KPI} für jede Teilhypothese identifiziert. Das Ergebnis dieser Analyse war, dass
die Datenquellen geeignet und ausreichend für die Realisierung eines \ac{BI} Prozesses sind. Im nächsten Schritt
wurden verschiedene Analyseverfahren aufgelistet, die für die Bearbeitung der Daten in Betracht kommen können.
Dazu wurden diese Verfahren kurz beschrieben und mögliche Beispiele und Ansätze innerhalb der Teilhypothesen
identifiziert. Dadurch ist feststellbar, dass die Analysemethoden für den Fall von deskriptiver
\ac{BI} ausreichen. Bei Vorhersagemodellen ist eine Beantwortung zu diesem Zeitpunkt noch nicht möglich. Im folgenden Kapitel
(Kapitel \ref{toc:planungeinesbiprozessesfuereinminingrechenzentrum}) wurde auf Grundlage der bisher erarbeiteten
Ergebnissen eine Fallstudie durchgeführt, die innerhalb der Genesis Group stattfand. Diese Fallstudie
prüft die innerbetriebliche Umsetzbarkeit der bisherigen Ergebnisse und ergänzt diese. Dazu wurde im ersten Schritt
eine Stakeholderanalyse durchgeführt. Diese wird für die Planung eines innerbetrieblichen \ac{BI} Prozesses
benötigt. Folgend wurden die einzelnen Schritte des Prozesses analysiert und mögliche Softwareprodukte identifiziert.
Es wurde zudem ein Fokus auf die Präsentation der Daten gelegt, da dies noch nicht innerhalb der
vorhergehenden Argumentation besprochen worden ist. Schlussendlich wurde die Fallstudie analysiert und ein Ergebnis
für die Haupthypothese gebildet.

Auf Basis der Argumentation und der Fallstudie ist feststellbar, dass die Teilhypothesen und dadurch auch die Haupthypothese
wahr sind. Es ist möglich, dass Rechenzentren, die Mining von Kryptowährungen betreiben, durch die Anwendung eines \ac{BI}
Prozesses finanziell optimiert werden können.

Es existieren jedoch Limitationen bei dem gewählten Vorgehen innerhalb dieser Arbeit. Dazu zählen die folgenden Punkte:
\begin{itemize}
    \item Letztendlich ist der \ac{BI} Prozess in der vorliegenden Arbeit im theoretischen Kontext geplant worden. Ob ein
    solcher Prozess tatsächlich in der Realität funktioniert, kann nur geklärt werden, indem diese Planung in der Praxis
    umgesetzt wird. Die Wahrscheinlichkeit, dass das nicht funktioniert, ist jedoch sehr gering, da die Fallstudie
    die praktische Umsetzung bestmöglich abdecken soll.
    \item Die verschiedenen Formen der Analyse wurden im Laufe der Arbeit aufgezeigt. Jedoch wurde auf eine
    konkrete Ausarbeitung dieser Analyseformen und deren Implementierung verzichtet. Es kann daher nur eine Umsetzung der
    Analysealgorithmen aussagen, ob diese den Anforderungen genügen.
    \item Die Integration eines \ac{BI} Prozesses wurde nur anhand der vorliegenden Datenquellen besprochen. Bei dem Fehlen
    oder bei Existenz zusätzlicher Datenquellen kann sich die Gestaltung des \ac{BI} Prozesses in einer Art und Weise
    ändern, wie es nicht in dieser Arbeit besprochen wurde. Allerdings sind die aufgezeigten Datenquellen mit einer
    hohen Wahrscheinlichkeit in jedem Unternehmen verfügbar, das sich am Mining beteiligt. Auch bei Verwendung anderer
    Typen von Mining Hardware können sich die Parameter ändern, die zu einer Änderung des \ac{BI} Prozesses führen.
    \item Andere Konsensalgorithmen als \ac{PoW} wurden in dieser Arbeit nicht besprochen. Diese müssen individuell angepasst
    in einen \ac{BI} Prozess überführt werden.
\end{itemize}

Trotz dieser Punkte ist es möglich die Haupthypothese als beantwortet anzusehen, da alle Datenquellen, die in dieser
Arbeit verwendet worden sind, in jeglichen Unternehmen am Markt verfügbar sind. Nur in Spezialfällen müssen
kleine Anpassungen vorgenommen werden, die jedoch nicht die Gesamtaussage dieser Arbeit verändern werden.

\subsection{Further research approaches} \label{toc:ausblick}

Aus den Limitationen, die in Kapitel \ref{toc:ueberpruefungderhypothesen} identifiziert wurden, können weitere
Forschungsansätze abgeleitet werden. Diese Arbeit hatte den Fokus auf \ac{BI} im Umfeld der Bitcoin Blockchain und deren
relevanten Technologien. Jedoch existieren auch andere Anwendungsfälle der Blockchain Technologie, deren Untersuchung im
\ac{BI} Kontext interessant sein kann. Dazu zählen unter anderem die folgenden Technologien:
\begin{itemize}
    \item \textbf{\ac{DeFi}: }Im Zuge dieser Arbeit wurden Tauschbörsen als Datenquelle verwendet. Diese sind von Natur
    aus zentralisierte Dienstleistungen von einzelnen Unternehmen, wie beispielsweise Binance\footnote{https://www.binance.com/en}
    oder Kraken\footnote{https://www.kraken.com}. Entwickler arbeiten bereits an der dezentralen Entsprechung dieser
    Dienstleistungen, die unter dem Schlagwort \ac{DeFi} genannt werden.\footcite[Cf.][]{wiesflecker2021trends} Auch wird
    durch diese Technologie ein Kreditwesen im Bereich der Kryptowährungen eingeführt, was gerade für Unternehmen interessant
    sein kann, die Liquidität bei Kryptowährungen benötigen.\footcite[Cf.][]{wiesflecker2021trends} Die
    Erforschung der Rolle von \ac{DeFi} im Bereich \ac{BI} ist daher ein aktuelles Beispiel einer auf diese Arbeit
    aufbauenden Forschungslücke.
    \item \textbf{Smart Contracts: }Seit einiger Zeit werden auch sogenannte Smart Contracts mit Hilfe der Blockchain
    Technologie abgebildet. Das bekannteste Beispiel für die Umsetzung ist die Ethereum
    Blockchain.\footcite[Cf.][]{wiesflecker2021trends} Smart Contracts sind dezentrale Entsprechungen klassischer Verträge
    innerhalb einer Blockchain.\footcite[Cf.][]{wiesflecker2021trends} Da gerade Verträge mit Kunden, Dienstleistern
    und Lieferanten interessant für \ac{BI} Systeme sind, wäre daher die Erforschung des dezentralen Pendants eine
    Lücke, in welcher noch Forschung betrieben werden kann.
    \item \textbf{\acp{NFT}: }Diese Tokens repräsentieren einmalige Werte.\footcite[Cf.][]{wiesflecker2021trends} 
    Dabei können solche Tokens beispielsweise relevante Investments von Unternehmen in Kunst, Instrumente
    und andere einmalige Anlagewerte repräsentieren. Wie und ob diese mit dem Bereich \ac{BI} zusammenspielen, kann im Rahmen
    einer weiteren Arbeit zu diesem Thema bearbeitet werden. 
    \item \textbf{Weitere Konsensalgorithmen und Währungen: }Wie eingangs zu dieser Arbeit beschrieben, wurden nur die Bitcoin
    Blockchain und der \ac{PoW} Konsensalgorithmus innerhalb dieser Arbeit analysiert. Da noch weitere Konsensalgorithmen,
    wie beispielsweise \ac{PoS}, existieren, wäre die Untersuchung weiterer Kryptowährungen und Konsensalgorithmen
    von Interesse. Durch eine solche Untersuchung kann diese Arbeit in geeigneter Art und Weise ergänzt werden und
    das Thema \ac{BI} und Kryptowährungen vertieft werden.
\end{itemize}

Aufgrund der Bewahrheitung der Haupthypothese dieser Arbeit können diese Aspekte die Einsatzmöglichkeiten von \ac{BI} im
Kryptomining Kontext vertiefen und weitere mögliche Anwendungsfälle identifizieren.
